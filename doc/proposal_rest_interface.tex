\documentclass[a4paper]{article}

\usepackage{booktabs}
\usepackage{fancyvrb}

\title{General REST interface for CSPA services}
\author{Jan van der Laan \and Edwin de Jonge}

\begin{document}

\maketitle

\section{Introduction}

In this report we propose a generic REST (REpresentational State Transfer)
interface, which could be adapted by a large number of services created in the
CSPA project. Many services are simple data processing steps. They receive one
or more datasets with some parameters.  They will perform some sort of
computation, which could take a substantial amount of time and, when finished,
they return one or more result data sets with some logging information.  Since
these services all follow the same pattern, it is possible to design a generic
REST interface for these services.

Furthermore, we believe it is possible to implement a wrapper implementing this
interface which can be used by many of the services (any service which consists
of a executable which can be run using some command line parameters). This would
reduce the time to implement a CSPA service considerably.  Furthermore, this
also means that institutes can work together on implementing and perfecting this
wrapper, creating a much more robust and tested interface. 

RESTful applications use HTTP requests for CRUD (Create, Read, Update and
Delete) operations and maps these on the HTTP verbs POST, GET, PUT and DELETE.
All state is modeled with resources. We consider the following as a resource:

\begin{itemize}
  \item Input data are \emph{data resources} to be retrieved by the service.
  \item Results are \emph{data resources} generated by the service.
  \item Log files are \emph{text resources} generated by the service.
  \item One processing step is modeled as a \emph{job resource} located at the
  service.
\end{itemize}

Therefore, a service is called by creating (posting) a job on the service.  A
service is called by creating a job on the service. This job contains references
to the data resources needed by the service. The service adds to this job
information on the status of the job and, when the job has finished, references
to the log files and result data resources. Each job has a unique URL where the
current status of the job can be retrieved (using a get). 


\section{REST interface}

\subsection{Job}

The key resource for the service is the job resource. A job resource contains
the following information:
\begin{description}
  \item[id] Job id to be used in querying the job properties.
  \item[url] Unique URL to this job (includes job id).
  \item[name] Human readable name of the calling process.
  \item[version] Version of the service that created this job.
  \item[input] The input is service specific and can consist of:
    \begin{itemize}
      \item Data resources: URL endpoints to input data, that are to be
      retrieved using GET by the service
      \item Configuration parameters. Simple types (such as string, integer or
      float) can be passed by value. Complex configuration object can be passed
      by reference (URL). 
    \end{itemize}
  \item[result] The results are service specific and will consist of:
    \begin{itemize}
      \item Data resources: URL endpoints to generated output data that are to
      be serviced by the service.
    \end{itemize}
  \item[log] Contains a URL to the log file of the job. 
  \item[status] Can have one of the following values: `created', `scheduled',
  `running', `finished', `error'.
  \item[created] Time at which the job was created (in UTC).
  \item[started] Time at which the job was started (in UTC)
  \item[finished] Time at which the job has finished (in UTC)
  \item[on\_end] A user can optionally provide a callback URL which gets called
  (post of the job URL) when the job has finished. 
\end{description}
Some of this information is only available when relevant. For example, the
\emph{result} of the job resource is only available when the job has finished
(and the job \emph{status} is `finished'). 

The service returns the job in JSON (JavaScript Object Notation) format. An
example of a complete job description as could be returned by a service (we used
the Linear Rule Checking service as an example):
{\small
\begin{Verbatim}
{
  "id" : "1234",
  "url" : "http://example.com/LRC/job/1234",
  "version": "0.0.1",
  "name" : "my_process",
  "status" : "finished",
  "input" : {
    "data" : {
      "type" : "ddi+csv",
      "meta" : "http://previous/service/job/1/result/output/meta",
      "data" : "http://previous/service/job/1/result/output/data",
    },
    "rules": {
      "type" : "text",
      "data" : "http://allthedatayouneed.com/myrules.txt"
    }
  }, 
  "result": {
    "checks" : {
      "type" : "ddi+csv",
      "meta" : "http://example.com/LRC/job/1234/result/checks/meta",
      "data" : "http://example.com/LRC/job/1234/result/checks/data"
    }
  }, 
  "log": {
    "type":"text"
    "url" : "http://example.com/LRC/job/1234/log",
  },
  "created": "2014-01-01T12:00" ,
  "started": "2014-01-01T12:00" ,
  "finished": "2014-01-01T12:05" ,
  "on_end" : "http://callback.com"
}
\end{Verbatim}
}
One thing that can be noted is that the references to the data resources consist
of (two or) three parts: type, data and meta. The exact method of passing data
around in CSPA is currently undecided. Section~\ref{sec:dataresources} further
discusses this topic. For the REST interface the exact method is not important.
The only thing that is important is that data is passed by reference, e.g. by
passing URLs to data resources. 

When creating a new job, only a subset of the information above needs to be
passed to the service, namely: \emph{name}, \emph{input} and optionally
\emph{on\_end}. The service expects the job in JSON format, for example:
{\small
\begin{Verbatim}
{
  "name" : "my_process",
  "input" : {
    "data" : {
      "type" : "ddi+csv",
      "meta" : "http://previous/service/job/1/result/output/meta",
      "data" : "http://previous/service/job/1/result/output/data",
    },
    "rules": {
      "type" : "text",
      "data" : "http://allthedatayouneed.com/myrules.txt"
    }
  }, 
  "on_end" : "http://callback.com"
}
\end{Verbatim}
}

\subsection{Description of the interface}

Table~\ref{interf} gives an overview of the proposed REST interface.  A service
can be invoked by posting a job description to \texttt{/<servicename>/job}. This
will create the job and schedule it for execution. On successful creation of the
job a URL to the newly created job is returned. These URLs have the following
form: \texttt{/<servicename>/job/<jobid>}.  Using this URL information on the
job can be requested and the job can be deleted.

\begin{table}
  \caption{REST interface for CSPA services}
  \label{tab:interf}
  \begin{tabular}{l l l p{0.6\textwidth}}
    \toprule
    Resource & HTTP & Respone & Description \\
             & verb & code    &  \\

    \midrule
    \multicolumn{4}{l}{\texttt{/<servicename>/job}} \\
    & POST &  & Creates a job. Uses \emph{input parameters package} \\
    &   & 201 & Job successfully created. Returns URL to created job. Job execution is started. \\
    &   & 412 & Precondition failed. Job is not created.\\
    & GET  &  & Returns a list of all jobs at the service \\

    \multicolumn{4}{l}{\texttt{/<servicename>/help}} \\
    & GET &   & Returns human-readable help describing input and output parameters of the service. \\
    &   & 200  & OK \\

    \multicolumn{4}{l}{\bf\texttt{/<servicename>/example}} \\
    & GET &   & Returns a working test example for the service \\
    &   & 200  & OK \\

    \multicolumn{4}{l}{\bf\texttt{/<servicename>/example/input/<datasetname>}} \\
    & GET &   & Returns a working test example data set for the service \\
    &   & 200  & OK \\
    &   & 404  & NotFound \\

    \multicolumn{4}{l}{\texttt{/<servicename>/job/<jobid>}} \\
    & GET &    & Returns \emph{information response package} regarding specific job \\
    &   & 200  & OK \\
    &   & 404  & NotFound \\
    & DELETE & &  \\
    &   & 200  & OK \\
    &   & 401  & Unauthorized \\

    \multicolumn{4}{l}{\texttt{/<servicename>/job/<jobid>/result/<datasetname>/data}} \\
    & GET &    & Returns the physical data set \\
    &   & 200  & OK \\
    &   & 204  & NoContent. The data set is incomplete and is not returned. \\
    &   & 404  & NotFound \\

    \multicolumn{4}{l}{\texttt{/<servicename>/job/<jobid>/result/<datasetname>/meta}} \\
    & GET &    & Returns the meta data of the data set \\
    &   & 200  & OK \\
    &   & 204  & NoContent. The data set is incomplete and is not returned. \\
    &   & 404  & NotFound \\

    \multicolumn{4}{l}{\texttt{/<servicename>/job/<jobid>/log}} \\
    & GET &    & Returns the log file \\
    &   & 200  & OK. Returns the complete log file \\
    &   & 206  & PartialContent. Returns an incomplete log file. \\
    &   & 404  & NotFound \\
    \bottomrule
  \end{tabular}

\end{table}



\subsection{Chaining services}


\section{Data resources}
\label{sec:dataresources}



\section{Conclusion}

\begin{itemize}
  \item The exact contents of the job resource
\end{itemize}






\end{document}
